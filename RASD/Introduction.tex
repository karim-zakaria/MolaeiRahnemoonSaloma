\section{Introduction}

\subsection{Purpose}
\subsubsection{General Purpose}
Nowadays, an ever-increasing number of cars and a shortage in the number of police officers caused the emergence of various traffic violations and accidents. Although two traditional solutions to solve these problems were rising the number of police officers and their equipment, due to the poor efficiency and inordinate cost, it is not feasible to continue this trend. This is where the power of technology can take the responsibility to help authorities to bring the order to the streets.

The only solution to assist authorities without expanding budgets is to participate in people with an intuitive and simple method. Hence, the \emph{SafeStreets} app is proposed, which provides the possibility of reporting traffic violations and accidents by taking advantage of crowed-sourcing. Users can report violations by just taking pictures of infringement and license plate, then sending them.
\subsubsection{Goals}
The goals that the system aimed to achieve are presented as follows:
\begin{itemize}
\item  \goal{1}
\item  \goal{2}
\item  \goal{3}
\item  \goal{4}
\item  \goal{5}
\item  \goal{6}
\end{itemize}

\subsection{Scope}
The \emph{SafeStreets} system shall be providing four main functions to various users; in this section, the system boundaries and scope used to define the limitations and different responsibilities of the S2B. 

The first of the main functionalities is the enabling of users to report traffic violations. Regarding this, some phenomena are regarded as world phenomena not viewed by the system due to its limitations such as the fact that the system does not directly detect a violation. However, it can be accounted for by the system through a traffic report made by the users. Moreover, another functionality that has to do with the users is the publishing of collected data to be viewed by the users in a refined representation to help them consider the safety of various areas based on traffic violations. The data is also communicated to the authorities but with different levels of details.

The other two main functions have to do with the \emph{SafeStreets} system providing services to government authorities. The domain limitations of the system affecting this interaction are also discussed in this section. Such as, the fact that the system is only able to make suggestions for preventive measures to the authorities based on the accident data that have been communicated. Meaning, that the system does not have any knowledge of accidents unless they are reported by the authorities and that the system can only suggest interventions and neither put them into place nor can detect them being applied. Moreover, a second function to the authorities would be the communication of traffic reports received from users to be later used by government officials to give out traffic tickets, the system responsibilities to support this process is to prevent the users from tampering with images \emph{digitally} and to provide the collected reports to the authorities proactively. In other words, physical tampering with license plates to mislead authorities and the actual process of giving out tickets is not part of the application domain. 

Below is a table summarizing and classifying the different phenomena that are related to the system functionalities. 
Main system functionalities:
F1: Reporting of violations
F2: Communication of collected data to users
F3: Suggestion of interventions
F4:  Communication of reports for ticketing


\begin{table}[hbtp]
\footnotesize
\centering
\settowidth\tymin{\textbf{Functionality}}
\setlength\extrarowheight{2pt}
\begin{tabulary}{\textwidth}{|C|C|C|C|}
\hline
\textbf{Phenomena} & \textbf{Classification} & \textbf{Justification} & \textbf{Functionality} \\ 
\hline 
Physical tampering with license plate &
World & 
Pure world phenomena since no measures are to be applied to detect nor prevent this phenomenon therefore it is unobserved by the system & 
F1, F4
\\ \hline
Issuing of tickets & 
World & 
The actual issuing of the tickets is the responsibility of the authorities the system has no part in it and does not have access to the data regarding issued tickets & 
F4
\\ \hline
Putting preventive measures for traffic violations into place & 
World & 
The application of preventive measure by the municipality is also a pure world phenomenon as the system has no means of knowing new measures by applied & 
F3
\\ \hline
Traffic violations & 
World & 
The system does not directly observe or detect committed traffic violations if they are not reported by the user then the system cannot be held responsible for not having knowledge of them & 
F1, F4
\\ \hline
Occurrence of accidents & 
World & 
Similarly, to traffic violations unless system acquires this kind of data through the authorities it has no way of detecting such phenomena & 
F3
\\ \hline
Publishing of insights regarding the accumulated data & 
Shared & Performed by the machine observed by users and authorities in the world world & 
F2
\\ \hline
Reporting traffic violation & 
Shared & 
Performed by users in the world observed by the machine & 
F1, F4
\\ \hline
Publishing of accident data by the municipality & 
Shared & 
Performed by authorities in the world observed by the machine & 
F4
\\ \hline
Suggesting interventions & 
Shared &
Performed by system and communicated to authorities then if applied observed by the world & 
F3
\\ \hline
Municipality report submission & 
Shared &
Performed by system and communicated to authorities therefore it is observed by the world & 
F4
\\ \hline
\end{tabulary}
\caption{\label{tab:world-machine-phenomena}World and machine phenomenas.}
\end{table}



\subsection{Definitions,Acronyms,Abbreviations}
	example text


\subsection{Revision history}
	example text


\subsection{Reference Documents}
Systems and software engineering — Life cycle processes — Requirements engineering IEEE 29148

Alloy Documentation
UML


\subsection{Document Structure}
The requirement analysis and specification document(RASD) is consisting of five chapters, an outline is presented as follows:

\textbf{Chapter 1} is an introduction, which states the problems of traffic violations and also describes the previous unsuccessful methods that did not have tangible results. In addition to that, it illustrates the goals that the system aimed to reach and defines the scope of the system by pointing the world and shared phenomena.

\textbf{Chapter 2} is about the perspective of the product along with details in actors, shared phenomena, assumptions, dependencies and constraints. On top of these for more elaboration class and state diagrams are used to demonstrate the general view of entities and behaviors of actors, respectively.

\textbf{Chapter 3} contains information on interface requirements in terms of software, hardware and communication between them. Furthermore, user system interactions are represented in detail through use case diagram, sequence diagram and scenarios. Moreover, most significantly functional requirements with associated domain assumptions are declared, followed by design constraints and system attributes.

In \textbf{chapter 4}, Alloy which is a declarative specification language used to model behaviors and analyze various parts of the system.

Chapter 5 indicates the effort spent on each part of the document for each group member.


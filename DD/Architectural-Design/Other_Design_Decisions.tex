\subsection{Other Design Decisions}
\subsubsection{Technologies and Platforms}
To begin with selected programming languages for the implementation, the backend part of the system is going to be implemented using the  \emph{Flask web application framework}. There are a set of technical reasons for this decision however important ones are mentioned in the following. Flask allows us to develop our backend as fast as possible because of its simplicity and rich documentation, besides that, it’s flexible for providing  \emph{REST APIs} based on our desired functionalities. Additionally, the flask is a lightweight framework compared to Django in many terms, in particular enabling developers to create a service-driven system which means to meet users’ needs thoroughly and completely. Apart from this, flask comes with the flask-SQLAlchemy which is a bundle of SQLAlchemy for this framework. SQLAlchemy is an ORM for the python programming language, by using ORM we get several advantages namely, shielding application against SQL injection through filtering data, having an abstract concept of the database that makes database migration easier and enabling easy updating, maintaining and data manipulation. Last but not least, considering flask for backend gives us the possibility of having invaluable numbers of python modules.
Furthermore, for the client-side implementation we opted to target Android platform, the foremost reason is based on the statistics which indicates that Android platform accounts for more than 60 percent of the market share of mobile operating system among users. In order to develop our mobile application, we will use Java programming language which is the native programming language for Android platform. This choice gives us the feasibility of developing high performance application by taking advantage of rich libraries. Moreover, we are going to use Android SDK version 21 which is for Android 5.0(Lollipop) because Google has stopped supports for lower version than that.
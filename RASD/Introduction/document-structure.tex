The requirement analysis and specification document(RASD) is consisting of five chapters, an outline is presented as follows:

Chapter 1 is an introduction, which states the problems of traffic violations and also describes the previous unsuccessful methods that did not have tangible results. In addition to that, it illustrates the goals that the system aimed to reach and defines the scope of the system by pointing the world and shared phenomena.

Chapter 2 is about the perspective of the product along with details in actors, shared phenomena, assumptions, dependencies and constraints. On top of these for more elaboration class and state diagrams are used to demonstrate the general view of entities and behaviors of actors, respectively.

Chapter 3 contains information on interface requirements in terms of software, hardware and communication between them. Furthermore, user system interactions are represented in detail through use case diagram, sequence diagram and scenarios. Moreover, most significantly functional requirements with associated domain assumptions are declared, followed by design constraints and system attributes.

In chapter 4, Alloy which is a declarative specification language used to model behaviors and analyze various parts of the system.

Chapter 5 indicates the effort spent on each part of the document for each group member.

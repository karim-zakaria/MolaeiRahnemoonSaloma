\subsection{Document Structure}
The Software Design Document(\textbf{DD}) is comprised of six main chapters. Which shall be described in this section of the document:

\textbf{Chapter 1 (Introduction):} provides an overview of the document as a whole; describing, the various sections constituting this document, as well as, the intended use of this document.

\textbf{Chapter 2 (Architectural Design):} a detailed description of the architecture to be developed during the implementation phase of the system; spanning from a high-level component view to a detailed run-time description of the different modules of the system. This is used as a guideline for the development team in order to have a clear idea of how the system should be built.

\textbf{Chapter 3 (User Interface Design):} an overview of the design of the different interfaces that the users of the system shall be interacting with the system through; in order to utilize the functionalities of the system according to their needs. This overview is concerned with the visual aspects of the user interfaces.

\textbf{Chapter 4 (Requirements Traceability):} provides a link between the design decisions in this document and the requirements of the system described in the \emph{Requirements and Specification Document}. This is done by providing an explanation of how the system design described in this document fully satisfies the requirements the system must abide by.

\textbf{Chapter 5 (Implementation, Integration and Test Plan):} describes the approach to be followed during the development and testing phase of the system. This is also provided as a clear guideline for the development team to follow.

\textbf{Chapter 6 (Effort Spent):} summarizes the efforts of the team members in developing this document in terms of time spent on each of the sections of the document.

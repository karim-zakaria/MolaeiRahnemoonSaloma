\subsection{Scope}
In this section, the scope of the system previously described in the \emph{Requirements and Specification Document} shall be revisited. As well as, consider some more in-depth aspects of the system.

As previously stated in the \emph{Requirements Document}, the \emph{SafeStreets} system shall be providing four main functions to various users; in this section, the system boundaries and scope used to define the limitations and different responsibilities of the S2B. 

The first of the main functionalities is the enabling of users to report traffic violations. Regarding this, some phenomena are regarded as world phenomena not viewed by the system due to its limitations such as the fact that the system does not directly detect a violation. However, it can be accounted for by the system through a traffic report made by the users. Moreover, another functionality that has to do with the users is the publishing of collected data to be viewed by the users in a refined representation to help them consider the safety of various areas based on traffic violations. The data is also communicated to the authorities but with different levels of details.

The other two main functions have to do with the \emph{SafeStreets} system providing services to government authorities. The domain limitations of the system affecting this interaction are also discussed in this section. Such as, the fact that the system is only able to make suggestions for preventive measures to the authorities based on the accident data that have been communicated. Meaning, that the system does not have any knowledge of accidents unless they are reported by the authorities and that the system can only suggest interventions and neither put them into place nor can detect them being applied. Moreover, a second function to the authorities would be the communication of traffic reports received from users to be later used by government officials to give out traffic tickets, the system responsibilities to support this process is to prevent the users from tampering with images \emph{digitally} and to provide the collected reports to the authorities proactively. In other words, physical tampering with license plates to mislead authorities and the actual process of giving out tickets is not part of the application domain. 

Moreover, in this document, some more technical issues regarding the functioning of the system need to be discussed. Primarily, the security aspect of the system; more specifically, data security.
Since the system will be dealing with the collection and communication of sensitive data while performing more than one of the main functions; the system must, at all times ensure the safe transmission and storage of data and the application of measures to prevent any means of data tampering. The data being discussed includes but is not limited to, user personal data and detailed data of traffic reports.

D1: User provided image of traffic violations is not tampered with physically.
D2: User location included in the report assumed to be the true unmodified location.
D3: Each plate number is unique and registered to only one vehicle.
D4: Each user’s fiscal code is unique.
D5: User devices used for reporting violations have a functioning camera and GPS.
D6: The communication of data regarding reported violations to the authorities is assumed to be done proactively by the system.
D7: The plate number of violating vehicles is readable and clearly visible in images included in the report.
D8: The municipality is assumed to provide an interface for the submission of records of reported violations, refined insights produced by the system and suggested interventions.

In order to use SafeStreet, users are assumed to have a smartphone or tablet with capabilities of connecting to the internet, taking a picture and having a GPS sensor. More in details, the connection of internet via WiFi, 3G or 4G is needed to sent reports to the system, to recognize license plate and indicating occurred violations, users need to take pictures and send them to SafeStreet's server, finally to locate where the violation is occurred GPS sensor is used to get the coordinates of the violation.

This system is aimed to provide service throughout Italy, the privacy of the users is significant for the system, Thereby it observes General Data Protection Regulation(GDPR) legislated by the European Union.


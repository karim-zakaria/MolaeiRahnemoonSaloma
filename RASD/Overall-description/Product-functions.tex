In the following subsection the various major functions of the \emph{SafeStreets} system; which were briefly mentioned in the \emph{Scope} section of the document. However, this section shall provide a more in depth description of said functions providing a broad yet clear overview of the system's functionality.

\subsection{Violation Report}
The \emph{Violation Report} function is the motivating main function of the \emph{SafeStreets} system. This function provided mainly to be used by the average or in other words civilian user entails the enabling of such users to formalize traffic violation reports and submit them to be later used by either the system internally or by an external benefactor; i.e., the municipality.
The described functionality is achieved through a user interface; ideally, a mobile phone through which the accesses the \emph{SafeStreets} interface and starts the process of reporting a violation. This process is briefly constituted by the imaging of the violation directly from the interface, the verification of the detected plate number and detected location and filling in some data regarding the details of the traffic violation, e.g., the violation type. Finally,  the user submits to report which is saved and used by multiple other functions as will be discussed in the coming sections.

\subsection{View Area Safety}
This second function is also mainly used by civilians; although, the data provided to users in this section are also provided to the authorities although through a different interface as shall be seen later. The \emph{View Area Safety} function is based on the previously mentioned \emph{Violation Report} functionality. Which shall be clear as we describe the exact details of this function.
In the \emph{View Area Safety} function the user would specify a certain area in which he is interested in knowing the safety of. Then, the user is presented with a representation of the safety of the specified area. This is done formalized using the aggregated reports from various users of the system and graphically represented using the aid of map API.

\subsection{Suggest Intervention}
The following function along with the last function are provided to different class of users or to be more specific benefactors of the \emph{SafeStreets} system. The described entity is the governing authorities in the different areas observed by the system through user reports. It is worth mentioning, that these two functionalities are achieved through a sort of mutual interaction between the \emph{SafeStreets} system and the municipality. In which a municipality must first provide a means of communication or interaction, i.e., an interface in order for the system to provide these services.
The \emph{Suggest Intervention} service is the process by which the system suggests possible interventions  to put into place by officials in certain areas; in order to decrease the number of violations which leads to accidents in such areas. To achieve this, the municipality must provide and interface to access the accident data in order to be coupled with the system's reports and also another interface to submit the suggestions.

\subsection{Violation Report Communication}
The last major function of the system is also one provided to the municipality. This service is simply the communication of the submitted traffic reports submitted by users while ensuring the secure transmission and ensuring that the data is not tampered with to a certain extent. This, can be done through the same interface which the system suggests possible interventions through. Moreover, the system could be provide more useful refined data to the municipality, such as, statistics on the different committed violations in various areas and most reported plate number.
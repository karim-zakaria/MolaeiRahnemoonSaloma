\subsection{Component View}
In the component diagram, four parts of the system are the API from other companies that provide some functionalities as service. The Google map API provides the map and Geo-location services for getting the map in the mobile phone app and injecting the data comes from the server. Firebase provides easy authentication for mobile users by push notification which uses the cloud messaging  REST API of this service, this will help to reduce the app expenses because we pay for what we used unlike any SMS API and it is way more easy to integrate into the android app and another side pros is avoid email or SMS spamming. Besides the storage  REST API  of the Firebase is used to store violation images. Storing the amount of image that the program is going to work with it needs too much space and handling a database specialized for multimedia usage, but as said previously, in Firebase we pay for what we used and for sure this method has more advantages to buy the storage beforehand.\\
Sentry system helps to log the system in certain parts and also makes a bug tracking mechanism for the problem happing in the android app and also server-side. Moreover, the database component is the representer of the database for each component of the system to simultaneously follow the microservice best practice and hold the scalability of the system. The components of the system will work with the database through the ORM used in the system, which is the SQLAlchemy. The authentication, storage, and map API are connected to the android app and server, which as shown in the diagram each of these interfaces has their own usage and the functionality of them is not the same. Furthermore, two authority API is assumed in the system because in one of them the manipulability sends the data for the system, and in another they the system will send the data related to violation reports and interventions by their request. The authorities usually work with a lot of APIs, and usually, they are not integrated into one system, therefore to make the system as compatible as possible with existing systems this architecture is chosen.\\
On the server-side, for more isolation, the system is broken into seven-part which help to easy deployment and further development in the system. There is no direct communication between the component and for the accessing the data related to one of these components from another one they will communicate like two separate systems, for example, the API provided by the "Registration\/Authentication and login" part which the ViewMe and ReportViolation part are using this API. In this case, the reasons to provide internal API instead of directed access by the functions to this subsystem is first to make the system more scalable in future development; second, for security which this part has sensitive information about the user is better to be isolated from the whole system and has more strict rules to access to it. The reset of subsystems do their job without any knowledge about the other parts of the system, and there is no direct connection between the outside of the system and database. Even for the data which comes from outside first, the data will be stored in the database by responsible component for that request or response and then it can be used by other components by request. The component of the system will be described in the following:\\

The ReportViolation will handle the functionalities related to reporting a violation or editing or removing a report, after the request received by this component based on the request it will perform the required functionality. For submitting a report all data will be stored in the database of the after the component checks the user-ID by the request from registration and login component and will store the URLs provide ed for images by the firebase. For editing a request the component only allows the user to edit the type of violation and comment part. Finally, for removing a report will be removed from the database.\\

ReportSubmission is responsible for the request from authorities which after getting the request based on the type of the report will send a request to the report violation part or intervention recommender then when the report was ready it will store the data related to the report in its databse. \\
ViewMe is part that allows users to change the data of their profile and see their previous reports. This part of the system after getting the request will ask from report violation or register and login components based on the requests to get the data related to each part and for other type requests like edit or remove will do the same. The reason for this component in the backend is the one-to-one mapping of the part from client-side to server-side to provide an easy path for future development. \\
ViewSafety will manage the request for the data, the system needs to provide for showing safe areas on the map on the client-side. This component after getting the request will ask for the violation report from the report violation component and at the same time will send the request to the authority database to get the accidents, after all, will make a proper response to the request by collected data.
The "Registration\/Authentication and login" is in the head of user registration and credential of the users to work with the system. This component after getting the registration request will use the Firebase API to verify the use of authentication and also get the specific user-ID assigned to the user to store it in the database for furher functionalities.  Also, other components will send requests to this component to verify a user before sending data for the client-side. \\
There is two part of the system which they do not provide any service for the normal end-users one is the interventions recommender which will analyze the data and recommend the necessary intervention to authorities to decrease the number of violation and accidents for its functionalities this component will send requests for the report violation component. Next is the mechanism to check the images of the violation report from users and detect those who are manipulated and they are not original which if the image is manipulated will send a request to the report violation to flag the report. \\
In the android-side of the system, there is one important component, which is the plate number recognition which detects the plate number by images is taken by users and shows the plate number in pre-report. The last component in the android app is the local database, which is necessary to store some data about the user and its reports. Nevertheless, the android app has some other part which they just representer and can not assume them as a component.


\begin{sidewaysfigure}
\begin{figure}[H]
\caption{Component Diagram of whole system}
\label{fig:Class}
\centering
\includesvg[width=\textwidth, height=0.8\textheight]{"Component_Diagram-10"}
\end{figure}
\end{sidewaysfigure}


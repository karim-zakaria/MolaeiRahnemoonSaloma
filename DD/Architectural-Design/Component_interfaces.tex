\subsection{Component Interfaces:}

\begin{sidewaysfigure}
\begin{figure}[H]
\caption{End-Point API Diagram}
\label{fig:endpoint-api}
\centering
\includesvg[width=\textwidth, height=2\textheight]{"endpoit_api"}
\end{figure}
\end{sidewaysfigure}

The endpoint diagram of the API shows the interfaces of each component in the server-side and how can use the API. The API is divided into five main sections in which four of them will handle functionalities for the end-user and one is for authorities to get the report from the system. The \emph{users} path will handle the \emph{sign-in}, \emph{sign-up}, \emph{login} and \emph{edit} or \emph{remove} the account by the \emph{HTTP} command. The \emph{Get} command is used for the login because of the security reasons which is not secure to use \emph{POST} command for the password. The \emph{violation} path will be in charge of the \emph{submit}, \emph{edit}, and \emph{remove} each violation report. The username in this path is hashed string from Firebase after authentication to avoid username leak to avoid any 'Man-In-The-Middle' attacks which with plain username there is this possibility to submit an unoriginal report with other user's username. The \emph{ViewSafety} path works without a credential to let third-party developers to use it in other apps. This path will get the location and type of data are needed and return the information. Moreover, the \emph{ViewMe} part of the android app has some functionalities which will use the \emph{users} path of the API but the \emph{ViewHistory} which is the part of this section in android app will use this path to return the history of the reports by one user which this subpath also allows user to edit some parts of the report or remove a report, and obviously, this part also needs the hashed username. Finally, the authority path will handle the request from the different office of the municipality which by this path can get the aggregated report between two dates and also could get the recommended intervention between two dates.

\clearpage
\begin{sidewaysfigure}
\begin{figure}[H]
\caption{Class Diagram of the server-side}
\label{fig:Class-server}
\centering
\includesvg[width=\textwidth, height=2\textheight]{"Class_Diagram-10"}
\end{figure}
\end{sidewaysfigure}

The class diagram follows the separation of the system in the component diagram, hence there are seven subsystems with one auxiliary class. In the five subsystems, there is a class in which their name is finished with the  \emph{"model"}, these classes are used with ORM ( \emph{SQLAlchemy}) which makes the database API for the app. The  \emph{SQLAlchemy} will use these classes to maps these models to the database and make tables for them in  \emph{MySQL}. The reason for choosing the ORM to communicate with the database is to increase the speed of development and make the system more portable for the feature development in which the database could be changed or a  \emph{NoSQL} can be used besides the  \emph{MySQL}, therefore the ORM will make a layer of abstraction. However, the cost of easy adaptation of the future updates is slower query run which by optimization the  \emph{SQLAlchemy} the difference with the bare-metal approach is ignorable.\\
Based on both class diagram the input data will be checked in three stages one in the android app which the view of the app then in the REST API which the presenter layer and the last time in the model class before committing into the database and all the function started with  \emph{"isValid"} are for checking the fields of request in body of JSON and the JSON itself to avoid the  \emph{SQL injection} and  \emph{NoSQL injection} attacks or any sabotage of the API. The database model of the intervention recommender will store all the recommends to use them in the future as input and also send them for authorities whenever they want. Moreover, the database model of image manipulation detection will store all the data which are necessary for the next run of this model because this model is not realtime to avoid too much overhead on system and database in  \emph{violation report} request so this model runs time to time and need the last time run's details to start from right place in database and if the is problem in the images of a report the originality filed of the report will be set to false by the functions in  \emph{img\_manip\_dete} class, therefore, there is no new record for the manipulated images. Also, in the  \emph{report\_submission\_model} the details of the request which comes from authorities will be stored and then the system will respond to the request.\\
Those classes which do not have the  \emph{"model"} at the end of their name are in charge of the functionalities of each subsystem. The  \emph{ViewSafety} and  \emph{ViewMe} classes are just the consumer of the data and they will not make any new record so they do not have the database model class.\\
As shown in the diagram the  \emph{account\_model} is the main database of the system because of the main functionality of the system is required this model after this model the  \emph{report\_model} is another main model with the same reason which many parts of the system working by the data stored by this model. The  \emph{GeoCoordanite} class is the dependency class for the  \emph{reportViolatoin},  \emph{ViewSafety}, and  \emph{InterventionRecommender} which these three classes have some location related data this class will handle all the functionalities which are necessary for the system and is not part on any subsystem.\\
The system is implemented in the  \emph{Thin-client\/ Thick-server} so the android part of the system is lightweight and generally is just the view classes that will get the information from the API or it will send data for it. However, the android app has the  \emph{plate recognition} subsystem which will detect and show the plate number in the pre-report to the user by the images are taken by him\/her.  In this subsystem, the  \emph{PR\_record\_model} will store all the palate numbers by the model in the local database to allow offline submit of the violation besides all the data of the violation report and also to collect them for the future improvement of the detection model. As same as the server-side, the  \emph{Geo-coordinate} will handle all the functionalities related to location and  \emph{ReportViolation} and  \emph{ViewSafety} have a dependency to this class. Also, each class will handle its communication to the server and there is no message router to avoid a breakpoint or bottleneck in the system in which the same approach is being followed by the server.\\


\begin{sidewaysfigure}
\begin{figure}[H]
\caption{Class Diagram of the client-side}
\label{fig:Class-client}
\centering
\includesvg[width=\textwidth, height=2\textheight]{"Class_Diagram_android-10"}
\end{figure}
\end{sidewaysfigure}

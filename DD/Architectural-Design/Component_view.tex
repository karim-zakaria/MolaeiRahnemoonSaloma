\subsection{Component View}
In the component diagram, there are four-part of the system which are the API from other companies that provide some functionalities as service. The google map API provides the map and Geo-location services and  \emph{Firebase} provides easy authentication for mobile users by push notification, plus the storage API for the storage of the violation images. Storing the amount of image that the program is going to work with it needs too much space and handling a database specialized for multimedia usage, so  \emph{Firebase} Storage is a more pertinent decision. In the same manner authentication by email or phone number (SMS) has it is own problems, so the application uses  \emph{Google Firebase} authentication. \emph{Sentry} system helps to log the system in necessary parts and also makes a bug tracking mechanism for the problem happing in the android app and also server-side. Moreover, the database it seld is another component based on the architecture of the system which should be more scalable, as a result, the system will work with the database through the API provided by the ORM used in the system, which is the  \emph{SQLAlchemy}. The authentication, storage, and map API are connected to the android app and server, which as shown in the diagram each of these interfaces has their own usage and the functionality of them is not the same. \\
Furthermore, two authority API is assumed in the system because in one of them the manipulability sends the data for the system, and in another they the system will send the data related to violation reports and interventions by their request. The authorities usually work with a lot of APIs, and usually, they are not integrated into one system, therefore to make the system as compatible as possible with existing systems this architecture is chosen.\\
 On the server-side, for more isolation, the system is broken into seven-part which help to easy deployment and further development in the system. The only connection between the subsystems is the API provided by the  \emph{"Registration\/Authentication and login"} part which the ViewMe and ReportViolation part are using this API. The reasons to provide internal API instead of directed access by the functions to this subsystem is first to make the system more scalable in future development; second, for security which this part has sensitive information about the user is better to be isolated from the whole system and has more strict rules to access to it.\\
  The reset of subsystems so their job without any knowledge about the other parts of the system, and there is no direct connection between the part outside of the system and database. Even for the data which comes from outside first, the data will be stored in the database by responsible component for that request or response and then it can be used by other components and no direct connection between the components. The  \emph{ReportViolation} will handle the functionalities related to reported violations, and the  \emph{ReportSubmission} is responsible for the request from authorities. Furthermore, the  \emph{ViewMe} is part which allows users to change the data of their profile and see their previous report and  \emph{ViewSafety} will manage the request for the data system needs to provide the safe areas on the map. The  \emph{"Registration\/Authentication and login"} is in the head of user registration and credential of the users to work with the system. There is two part of the system which they do not provide any service for the normal end-users one is the interventions recommender which will analyze the data and recommend the necessary intervention to authorities to decrease the number of violation and accidents, and next is the mechanism to check the images of the violation report from users and detect those who are manipulated and they are not original.\\
  In the android-side of the system, there is one important component, which is the plate number recognition which detects the plate number by images is taken by users and shows the plate number in pre-report. The last component in the android app is the local database, which is necessary to store some data about the user and its reports. Nevertheless, the android app has some other part which they just representer and can not assume them as a component.


\begin{sidewaysfigure}
\begin{figure}[H]
\caption{Component Diagram of whole system}
\label{fig:Class}
\centering
\includesvg[width=\textwidth, height=0.8\textheight]{"Component_Diagram-10"}
\end{figure}
\end{sidewaysfigure}

